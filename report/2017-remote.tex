\section{2017年国赛-远程幅频特性测试装置}
\subsection{器件配置}
\subsubsection{AD9854}
在本题中我们需要使用信号源对待测网络进行扫频,根据频率和相应采集到的电压进行计算,
因此将AD9854配置为单频模式,方便灵活控制频率;由于输出接到后级运放,因此不需要OSK功能。
同时为避免高频时出现欠采样,启用PLL并设置到4倍频
(板上晶振$\SI{30}{MHz}\times 4 = \SI{120}{MHz}$)

在程序里如下配置:

\begin{cbox}{ad9854.c}
  /* 外部更新 */
  ad9854_set_bits(ad9854_updclk, ad9854_updclk_external);
  /* 单频模式 */
  ad9854_set_bits(ad9854_mode, ad9854_mode_single);
  /* 旁路反Sinc滤波器 */
  ad9854_set_bits(ad9854_invsinc_byp, 1);
  /* 关闭输出键控 */
  ad9854_set_bits(ad9854_osk_en, 0);
  /* 启用PLL */
  ad9854_set_bits(ad9854_pll_bypass, 0);
  /* 200MHz以上需设为1 */
  ad9854_set_bits(ad9854_pll_range, 1);
  /* 4倍频,经测试至少能稳定输出至50MHz而不出现欠采样 */
  ad9854_set_bits(ad9854_pll_mult, 4);
\end{cbox}

在测试阶段尝试过10倍频($\SI{30}{MHz}\times 10 = \SI{300}{MHz}$,
即AD9854最大可设置的SYSCLK频率),但会出现功耗过大的问题,
甚至偶尔会出现复位失败无输出的现象,因此综合起见,倍频在6以内比较合适。